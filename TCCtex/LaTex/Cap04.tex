\section{DISCUSS\~AO}

\bigskip

{
Uma quest\~ao relevante que pode surgir a quem est\'a desenvolvendo um TCC e toma conhecimento da exist\^encia do TCCTeX
\'e: ``Por que eu deveria utilizar o TCCTeX?{\textquotedbl} Essa \'e uma\ quest\~ao pertinente tendo em vista que todo
estudante universit\'ario j\'a teve algum contato com os softwares editores de texto do tipo WYSIWYG (What You See Is
What You Get, ou O que voc\^e v\^e \'e o que voc\^e obt\'em) e sem duvida j\'a fez diversos trabalhos, atividades e
diversos tipos de textos nesse tipo editor de texto.}

{
A resposta para essa pergunta foi a nossa grande preocupa\c{c}\~ao ao desenvolver nosso projeto. O que faria um
usu\'ario de programas como Microsoft Word ou\ LibreOffice Writer deixar de trabalhar exclusivamente com esses
programas e come\c{c}ar a utilizar o TCCTeX?}

{
O que levaria uma pessoa que esta empenhada em desenvolver seu TCC desviar o seu foco do desenvolvimento do mesmo para
dedicar um pouco de seu tempo a aprender a utilizar o TCCTeX? Se o per\'iodo para\ o desenvolvimento de um TCC \'e, na
grande maioria dos casos, apertado, o fato de ter que se familiarizar com um novo software n\~ao diminuiria o tempo do
aluno dedicado ao TCC?\ }

{
Na nossa vis\~ao a quest\~ao correta a ser levantada \'e: ``O TCCTeX \'e realmente capaz de me fazer economizar tempo
durante a documenta\c{c}\~ao do meu projeto de TCC?'' A resposta para essa pergunta \'e: Sim. Na realidade o tempo
gasto para aprender a utilizar e se familiarizar com o TCCTeX \'e sem duvida muito menor que o tempo gasto para colocar
um TCC perfeitamente dentro de todos os padr\~oes estabelecidos pela ABNT (Associa\c{c}\~ao Brasileira de Normas
T\'ecnicas) e exigidos pela Universidade Metodista de S\~ao Paulo.}

{
Cada p\'agina de um trabalho acad\^emico deve estar dentro de padr\~oes muito bem detalhados e cada uma das etapas da
documenta\c{c}\~ao possuem diferen\c{c}as de formata\c{c}\~ao entre si, isso sem contar com as in\'umeras regras para
se colocar tabelas e figuras e suas listas.}

{
Mas o problema de colocar um trabalho nos padr\~oes estabelecidos utilizando os processadores de texto altamente
difundidos pelo mercado n\~ao est\'a s\'o na grande quantidade de pequenos detalhes que devem ser atentados nessa
tarefa, mas tamb\'em a grande irregularidade e inconsist\^encia desses programas, especialmente no que diz respeito \`a
formata\c{c}\~ao de grandes documentos.}

{
No desenvolvimento de um grande documento de texto diversas vezes o autor sente a necessidade de modificar a
organiza\c{c}\~ao do mesmo, seja mudando a ordem dos cap\'itulos,\ inserindo ou at\'e mesmo removendo par\'agrafos, mas
quando se trabalha com\ processadores de texto tarefas simples como essas podem se tornar dor de cabe\c{c}a. O simples
ato de alterar o texto de uma p\'agina pode refletir negativamente na formata\c{c}\~ao de todas as p\'aginas
subsequentes. E quando o autor deixa para formatar o documento ap\'os estar com todos os textos completos pode levar
muitas horas\ e deixar muitas falhas no caminho, principalmente para documentos muito grandes.}

{
Vale lembrar que at\'e mesmo manter um padr\~ao \'unico da primeira folha do primeiro cap\'itulo at\'e a \'ultima folha
do \'ultimo cap\'itulo de um grande documento pode se tornar uma tarefa \'ardua quando se trabalha exclusivamente com
processadores de texto WYSIWYG.}

{
A grande vantagem que o aluno pode obter ao documentar seu TCC com o auxilio do TCCTeX \'e justamente n\~ao ter que se
preocupar com a formata\c{c}\~ao do documento. Ap\'os aprender a utilizar o TCCTeX o seu foco poder\'a estar voltado
apenas para o conte\'udo do documento, sem ter que se preocupar com margens, \'indices, listas ou legendas.}

{
Acreditamos que o usu\'ario conseguira se familiarizar com o TCCTeX rapidamente devido a simplicidade da interface e a
grande quantidade de dicas de como utilizar o software que ser\~ao disponibilizadas\ no manual. O aluno facilmente
aprender\'a a utilizar o TCCTeX no dia a\ dia.}


\bigskip


\bigskip


\bigskip

