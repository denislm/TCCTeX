\chapter[INTRODU\c{C}\~AO]{\foreignlanguage{portuges}{INTRODU\c{C}\~AO}}

\bigskip

{
Nos dias atuais software faz parte da nossa vida, eles est\~ao em todos os lugares, suas fun\c{c}\~oes s\~ao as mais
diversas, sua utilidade incalcul\'avel. Ainda neste tema de\ software, notamos a dificuldade dos alunos na hora de
adequar a reda\c{c}\~ao de seus trabalhos ao padr\~ao ABNT; por outro lado professores dispendem horas corrigindo a
formata\c{c}\~ao do trabalho, nesse cen\'ario, o projeto, TCCTeX, se prop\~oe a auxiliar o estudante a desenvolver seu
TCC em etapas pr\'e-definidas, padronizando-o segundo a norma ABNT Metodista e gerando automaticamente o documento no
formato PDF, ap\'os ser processado em {\LaTeX}.}

{
O principal objetivo \'e desenvolver um software (TCCTeX) que \ recebe dados e arquivos em formato comum ao usu\'ario,
faz um processamento destes e como sa\'ida, produz um documento completo no padr\~ao {\LaTeX} seguindo as normas ABNT
Metodista, assim agilizando o processo de padroniza\c{c}\~ao do trabalho feito pelo aluno e tornando desnecess\'aria a
corre\c{c}\~ao da padroniza\c{c}\~ao feita pelo professor.}

{
Durante o desenvolvimento do nosso projeto, utilizaremos diversas ferramentas de apoio ao desenvolvimento adequado ao
nosso tipo software, entre elas Latex, NetBeans IDE. O desenvolvimento deste projeto utiliza\ o consagrado {\LaTeX}, um
produtor de texto acad\^emicos e matem\'aticos; Extreme Programming (XP), uma metodologia \'agil para desenvolvimento
de software que encoraja a colabora\c{c}\~ao entre as partes interessadas de um projeto de software; e NetBeans IDE, um
ambiente de desenvolvimento muito utilizado pelo mercado.}

{
A motiva\c{c}\~ao do grupo \'e a possibilidade de aprender a lidar com metodologias, t\'ecnicas e tecnologias com grande
potencial de mercado as quais esperamos poder ser um diferencial no desenvolvimento da nossa carreira. E mais a
oportunidade de desenvolver um software simples que ser\'a \'util a n\'os e a qualquer outro estudante que necessite
seguir as normas da ABNT Metodista para o desenvolvimento de um documento de TCC.\ }

{
Espera-se que este software possa contribuir\ para ajudar aos alunos da Escola de Engenharias, Tecnologia e
Informa\c{c}\~ao e futuramente qualquer outra faculdade da Metodista, no seu processo de constru\c{c}\~ao do TCC,
ganhando tempo, qualidade e para alunos e professores.}


\bigskip

