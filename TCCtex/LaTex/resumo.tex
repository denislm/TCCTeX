\begin{resumo}
{
\ A ideia por tr\'as de computadores digitais pode ser explicada dizendo que estas m\'aquinas t\^em a inten\c{c}\~ao de
realizar qualquer opera\c{c}\~ao que pode ser realizada por um computador\ humano (Alan Turing, 1950, tradu\c{c}\~ao
nossa). Na era da informa\c{c}\~ao e automa\c{c}\~ao, podemos mais do que nunca, moldar o mundo a nossa volta,
de\ cadeiras de rodas com software de sintetiza\c{c}\~ao de voz, a celulares que encontram a loja que
deseja.\ Softwares est\~ao em\ todos os lugares, em sistemas embarcados nos avi\~oes, no forno de micro-ondas, nos
carros, talvez esta seja uma das inven\c{c}\~oes mais \'uteis j\'a criadas.\ Nesse contexto, pensamos em cada coisa que
pode ser automatizada, otimizada, padronizada, diante dessa realidade constatamos que poder\'iamos auxiliar alunos que
est\~ao passando por um importante processo acad\^emico, o conhecido e temido Trabalho de Conclus\~ao do Curso (TCC). A
cada ano professores e alunos usam grande parte do seu precioso tempo para escrever, revisar, refazer seu texto para
avalia\c{c}\~ao. \ Nossa proposta ambiciona a automatiza\c{c}\~ao de parte deste processo, nos propomos a desenvolver
um software, que recebe arquivos de texto como entrada, processa, e padroniza o documento final nos padr\~oes ABNT
Metodista.\ Para tal, utilizamos t\'ecnicas e ferramentas de desenvolvimento de software como, Extreme Programming, uma
metodologia de desenvolvimento \'agil de software; \ NetBeans IDE, um ambiente integrado de desenvolvimento; {\LaTeX},
um conjunto de macros para o programa de diagrama\c{c}\~ao de textos TeX, utilizado amplamente na produ\c{c}\~ao de
textos matem\'aticos e cient\'ificos, devido a sua alta qualidade tipogr\'afica.\ }


\bigskip

{
Palavras-chave: TCCTeX, TCC, desenvolvimento de software, XP, Extreme Programming, processador de texto, {\LaTeX},\ TeX,
odt, pdf.}


\bigskip

\end{resumo}