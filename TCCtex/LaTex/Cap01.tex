\chapter{REVIS\~AO BIBLIOGR\'AFICA}

\bigskip

{
Durante o processo de desenvolvimento do TCC pesquisamos diversos softwares que s\~ao utilizados para o desenvolvimento
da documenta\c{c}\~ao de TCC a fim de ter o\ conhecimento sobre essas ferramentas, suas vantagens e desvantagens em
rela\c{c}\~ao ao desenvolvimento do TCC.}

{
As principais ferramentas utilizadas para desenvolvimento da documenta\c{c}\~ao do TCC s\~ao editores de texto WYSIWYG
(What you see is what you get, em portugu\^es ``O que voc\^e v\^e \'e o que voc\^e obt\'em'') como o~Microsoft Word que
diferente do {\LaTeX} possibilita que o documento que est\'a sendo criado possa ser editado e visualizado ao mesmo
tempo com suas altera\c{c}\~oes, por meio de atalhos que s\~ao selecionados pelo mouse\ ou atalhos, enquanto no
{\LaTeX} todo desenvolvimento do documento \'e feito por meio de c\'odigos que s\~ao interpretados e apresentados
visualmente ao usu\'ario apenas ap\'os a convers\~ao do documento para PDF ou algum outro formato pr\'e estabelecido. A
grande desvantagem de editores WYSIWYG \'e a dificuldade de formata\c{c}\~ao de documentos muito extensos e de
abordagem cient\'ifica, nos quais \'e exigida uma formata\c{c}\~ao padronizada e cheia de detalhes. Outra desvantagem
\'e na cria\c{c}\~ao e formata\c{c}\~ao de f\'ormulas matem\'aticas complexas\ que s\~ao muito comuns em documentos
cient\'ificos.}

{
Por esses motivos, muitas universidades levaram a exigir que os alunos utilizassem o {\LaTeX} para o desenvolvimento do
TCC, pois ele possui f\'acil formata\c{c}\~ao de qualquer tipo de texto e \'e um programa padr\~ao que\ existe h\'a
anos e n\~ao varia muito a cada nova vers\~ao al\'em de ser gratuito.}

{
Notamos que hoje, o melhor programa que permite o desenvolvimento de um TCC dentro de todas as normas ABNT e todas as
funcionalidades exigidas para isso \'e o {\LaTeX}, por\'em ele n\~ao \'e um\ programa intuitivo e de f\'acil
aprendizagem, principalmente para pessoas que n\~ao tem grande familiaridade com computa\c{c}\~ao.}

{
Em cada universidade a utiliza\c{c}\~ao das normas ABNT para formata\c{c}\~ao de trabalhos acad\^emicos \'e feita de
forma diferente e seguindo como base o documento ``Manual de Apresenta\c{c}\~ao para Trabalhos Acad\^emicos'' (TIMB\'O,
2013) desenvolvido pela Universidade Metodista de S\~ao Paulo, notamos a complexidade das normas e a grande\ quantidade
de cuidados que devem ser tomados pelos alunos para fazer uma documenta\c{c}\~ao corretamente. Usaremos os requisitos
deste documento como base para desenvolver o nosso software que intermediar\'a a intera\c{c}\~ao do aluno que
desenvolve o TCC com o {\LaTeX} e\ auxiliar\'a na formata\c{c}\~ao deste documento segundo as normas ABNT adotadas pela
UMESP.}

